\chapter{RELATED WORK}
\thispagestyle{plain}

\label{RelatedWork}


\section{Experimentation in ABMs}
\label{sec:abmexp}
Experimental platform for messing around with ABMs: (Bourjot -- ``A platform for the analysis of artificial self-organized systems'' 2004) (relevant?)

ABMs have been used to study behaviors in biological systems... Domain specific work interested in system-level behavior: ant lane formation \cite{couzin2003sol}, marching locusts \cite{buhl2006dom}, fish schools \cite{parrish2002sof}. Most of these studies are qualitative in nature.

particle swarm optimization: emperical study \cite{shi1998parameter}; more general approach: \cite{van2006study}

\section{Prediction of System-Level Behavior}

physics-based control policy \cite{spears2004dpb} -- similar to our approach, but the models are tightly coupled to the domain. the system was designed with system-level models in mind. Algebraic inversion for inverse mapping is nice. Inspires the nonlinear regression approach in \fw.

Macroscopic models of swarm robot systems \cite{lerman2002mmf}\cite{lerman2005rpm} -- similar in motivation, but models the system in a more specific way (FSAs). Specific to systems in which agents can be modeled as FSAs. Our approach is more general, since it just looks at parameters.

%not entirely relevant -- Idea of system-level control.... robot swarms \cite{mclurkin2004srt} -- tightly coupled with domain, describes the behaviors as actions (not properties), splits behaviors into hierarchies.


\section{Inversion of Neural Networks}
Inversion of neural networks: 
\begin{itemize}
\item (A Linden and J Kindermann ``Inversion of multilayer nets'' 1989 -- optimization problem solved by gradient descent) 
\item (Bao-Liang Lu ``Inverting Feedforward Neural Networks using Linear and Nonlinear programming'' 1990 -- formulate the inverse problem as a nonlinear programming problem, a separable programming problem or a linear programming problem)
\item (S. Lee and R.M. Kill ``Inverse Mapping of continuous functions using local and global information'' 1989 -- iterative update towards a good solution)
\item (Michael I. Jordan work with robot arm ``Forward Models: supervised learning with a distal teacher'' -- asks the question, what configuration of the robot arm will yield this behavior?)
\end{itemize}
All of these are optimization techniques. They do not return an actual mapping. Also, some of the techniques are restricted to neural networks (and are thus not algorithm-independent).






\chapter{CONCLUSIONS AND FUTURE WORK}
\thispagestyle{plain}

\label{Conclusions}

% Summarize the work
   % Explained how to define system-level properties
   % Outlined the framework for learning a forward mapping
   % Outlined the framework and provided an algorithm for learning the reverse mapping
   % Demonstrated with emperical results that my approach is accurate and efficient.

% The research project reached its aims
   % what did I try to achieve?
      % The main goal of the \framework is to bridge the gap between agent-level parameters and system-level properties.

      %\item Domain independent: The design of \fw should minimize the amount of configuration that is needed for each new domain;
      %\item Algorithm independent: any regression algorithm should be able to be applied with \fw;
      %\item Accurate: \fw should generate accurate predictions and control suggestions; and
      %\item Fast for the user -- interactions with the models generated by \fw should require minimal computational time.

   % How did I achieve those goals?
      % Generating meta-models of the agent-based models
      % Previous approaches only returned one point, which gives a possible configuration, but not \fw!





\section{Future Work}

In the course of the research presented in this dissertation, a number of possible directions for future work have arisen.
Each of the following subsections discuss a possible subproblem in ABM meta-modeling or extensions to \fw that could produce interesting results.

\subsection{Adaptive Meta-models}

% hidden properties that can change

\subsection{Correlated System-level Properties}

% need to predict all system-level properties as one.

\subsection{The Reverse-mapping Approach as a General Algorithm}

% the SCI approach and the quad-tree method

\newpage
\pagestyle{empty}

\begin{center}
\vspace{0.1in}
\large{\bf ABSTRACT} \par  
\bigskip \bigskip
\end{center}

\begin{flushleft}
{\bf Title of Thesis:} \thesistitle\\
Donald P. Miner, PhD in Computer Science, 2010 \\
\begin{singlespace}
{\bf Thesis directed by:}{\hspace{2.5mm}} \parbox[t]{3in}{Dr. Marie desJardins, Associate Professor\\
Department of Computer Science and \\ Electrical Engineering}
\end{singlespace}
\end{flushleft}

\begin{singlespace}

An ABM is a computational model that simulates a complex system of many autonomous agents.
For example, ABMs have been used to model food chains, the spread of disease, traffic situations and flocking animals.
All of the behaviors in an ABM, from agent-level local interactions to system-level behaviors, emerge from these local interactions, which are governed by the individual agent programs.
The behavior of each individual agent in an agent-based model (ABM) is typically well understood because the agents' program has adjustable control parameters that directly modify its behaviors.
However, users of ABMs do not typically have an explicit model of how the values of agent-level control parameters affect the observed system-level behavior of the ABM.
Because of this lack of a model, the emergent behavior of an ABM is generally reduced to a repetitive process of guess-and-check: select new agent-level parameter values and then observe and evaluate the result.
That is, the controller of the system is forced to use the agent-level parameters, which only indirectly affect the system-level behavior.
This process is repetative.

This dissertation presents a domain-independent approach for modeling how changes in agent-level control parameters affect the system as a whole, and for using this model to enable a user to predict and control the behavior of the system \textit{at the system level}.
The approach is implemented as a learning-based framework, named the \framework,  for defining correlations between agent-level parameter values and quantitative system-level behavior.
Regression methods such as k-nearest neighbor, locally weighted scatterplot smoothing, and nonlinear least squares are used to predict system-level property values, given the system configuration.
The mappings from agent-configuration space to system-level behavior space are then inverted with a novel approach to produce inverted predictions that suggest system configurations, given desired system-level property values.
The effectiveness and flexibility of the \framework are demonstrated with experiments from four diverse ABMs implemented in NetLogo, a popular multi-agent programmable modeling environment.

\end{singlespace}

\par\vfil

